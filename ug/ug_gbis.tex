\begin{comment}
\documentclass[11pt]{article}
\usepackage{graphicx}
\usepackage{makeidx}
\usepackage{amssymb}
\usepackage{amsmath}
\usepackage[stable]{footmisc}

\newcommand{\NAMDCONFWDEF}[5]{%
%  \addcontentsline{toc}{subparagraph}{#1}%
  {\bf \tt #1 } $<$ #2 $>$ \index{#1 parameter} \\%
  {\bf Acceptable Values: } #3 \\%
  {\bf Default Value: } #4 \\%
  {\bf Description: } #5%
}


\begin{document}
\end{comment}

\section{Generalized Born Implicit Solvent}
\label{section:gbis}

Generalized Born implicit solvent (GBIS) is a fast but approximate method for calculating molecular electrostatics in solvent as described by the Poisson Boltzmann equation which models water as a dielectric continuum.
GBIS enables the simulation of atomic structures without including explicit solvent water.
The elimination of explicit solvent greatly accelerates simulations;
this speedup is lessed by the increased computational complexity of the implicit solvent electrostatic calculation and a longer interaction cutoff.
These are discussed in greater detail below.


%\subsection{Theoretical Background}
\subsection{Theoretical Background}
Water has many biologically necessary properties, one of which is as a dielectric.
As a dielectric, water screens (lessens) electrostatic interactions between charged particles. 
Water can therefore be crudely modeled as a dielectric continuum.
In this manner, the electrostatic forces of a biological system can be expressed as a system of differential equations which can be solved for the electric field caused by a collection of charges.


%\subsubsection{Poisson Boltzmann Equation}
\subsubsection{Poisson Boltzmann Equation}
The Poisson Boltzmann equation (PBE),
$$
\vec{\nabla} \cdot \left[ \epsilon (\vec{r}) \vec{\nabla} \Psi(\vec{r})  \right] = -4\pi \rho^f (\vec{r}) - 4 \pi \sum_i c_i^\infty q_i \lambda(\vec{r}) \cdot \exp \left[ \frac{-q_i \Psi(\vec{r})}{k_B T} \right]
$$
is a nonlinear equation which solves for the electrostatic field, $\Psi(\vec{r})$, based on
the position dependent dielectric, $\epsilon(\vec{r})$,
the position-dependent accessibility of position $\vec{r}$ to the ions in solution, $\lambda(\vec{r})$,
the solute charge distribution, $\rho^f(\vec{r})$,
and the bulk charge density, $c_i^\infty$, of ion $q_i$.
While this equation does exactly solve for the electrostic field of a charge distribution in a dielectric, it is very expensive to solve, and therefore not suitable for molecular dynamics.


%\subsubsection{Generalized Born}
\subsubsection{Generalized Born}
The Generalized Born (GB) equation is an approximation of the PBE.
It models atoms as charged spheres whose internal dielectric is lower than that of the environment.
The screening which each atom, $i$, experiences is determined by the local environment;
the more atom $i$ is surrounded by other atoms, the less it's electrostatics will be screened since it is more surrounded by low dielectric; this property is called one atom descreening another.
Different GB models calculate atomic descreening differently.
Descreening is used to calculate the Born radius, $\alpha_i$, of each atom.
The Born radius of an atom measures the degree of descreening.
A large Born radius represents small screening (strong electric field) as if the atom were in vacuum.
A small Born radius represents large screening (weak electric field) as if the atom were in bulk water.
The next section describes how the Born radius is calculated and how it is used to calculate electrostatics.

%\subsubsection{Generalized Born OBC}
\subsubsection{Generalized Born Equations}
In a GB simulation, the total electrostatic force on an atom, $i$, is the net Coulomb force on atom $i$ (from nearby atoms) minus the GB force on atom $i$ (also caused by nearby atoms):
$$
\vec{F}_i = \vec{F}_i^{Coulomb} - \vec{F}_i^{GB}.
$$
Forces are contributed by other nearby atoms within a cutoff.
The GB force on atom $i$ is the derivative of the total GB energy with respect to relative atom distances $r_{ij}$,
\begin{eqnarray}
\vec{F}_i^{GB}&=&-\sum_j \left[  \frac{d E_T^{GB}}{d r_{ij}}  \right]  \hat{r}_{ji} \\
&=&-\sum_j \left[  \sum_k \frac{\partial E_T^{GB}}{\partial \alpha_k}\frac{d \alpha_k}{d r_{ij}} + \frac{\partial E_{ij}^{GB}}{\partial r_{ij}}  \right]  \hat{r}_{ji} \\
&=&-\sum_j \left[  \frac{\partial E_T^{GB}}{\partial \alpha_i}\frac{d \alpha_i}{d r_{ij}}+\frac{\partial E_T^{GB}}{\partial \alpha_j}\frac{d \alpha_j}{d r_{ij}}+ \frac{\partial E_{ij}^{GB}}{\partial r_{ij}}  \right]  \hat{r}_{ji} \; .
\end{eqnarray}
where the partial derivatives are included since the Born radius, $\alpha$, is a function of all relative atom distances.
The total GB energy of the system is
\begin{equation}
E_T^{GB} = \sum_i \sum_{j>i} E_{ij}^{GB} + \sum_{i} E_{ii}^{GB} \; ,
\end{equation}
where $E_{ii}^{GB}$ is the Born radius dependent self energy of atom $i$, and the GB energy between atoms $i$ and $j$ is given by
\begin{equation}
E^{GB}_{ij} = - k_e D_{ij} \frac{q_i q_j}{f_{ij}} \; .
\end{equation}
The dielectric term~\cite{SRIN99} is
\begin{equation}
D_{ij} = \left( \frac{1}{\epsilon_p} - \frac{\exp{\left(-\kappa f_{ij}\right)}}{\epsilon_s} \right) \; ,
\end{equation}
and the GB function~\cite{STIL90} is
\begin{equation}
f_{ij} = \sqrt{r_{ij}^2 + \alpha_i \alpha_j \exp{\left(\frac{-r_{ij}^2}{4 \alpha_i \alpha_j}\right)}} \; .
\end{equation}
As the Born radii of atoms $i$ and $j$ decrease (increasing screening), the effective distance between the atoms ($f_{ij}$) increases.
The implicit solvent implemented in NAMD is the model of Onufriev, Bashford and Case~\cite{ONUF00, ONUF04} which calculates the Born radius as
\begin{equation}
\alpha_k = \left[ \frac{1}{\rho_{k0}} - \frac{1}{\rho_k}\textrm{tanh}\left(\delta\psi_k - \beta\psi_k^2 + \gamma\psi_k^3\right)\right]^{-1} 
\end{equation}
where
\begin{equation}
\psi_k = \rho_{k0} \sum_l H_{kl} \; .
\end{equation}
$H_{ij}$ is the piecewise descreening function~\cite{ONUF04,HAWK96,SCHA90}; the seven piecewise regimes are
\begin{equation}
\textrm{Regimes} = \left\{
\begin{array}{r l l}
\textrm{0} & r_{ij} > r_c + \rho_{js} &(\textrm{sphere}~j~\textrm{beyond cutoff})\\
\textrm{I} & r_{ij} > r_c - \rho_{js} &(\textrm{sphere}~j~\textrm{partially within cutoff}) \\
\textrm{II} & r_{ij} > 4\rho_{js} &(\textrm{artificial regime for smoothing})\\
\textrm{III} & r_{ij} > \rho_{i0} + \rho_{js} &(\textrm{spheres not overlapping})\\
\textrm{IV} & r_{ij} > \left| \rho_{i0} - \rho_{js} \right| &(\textrm{spheres overlapping}) \\
\textrm{V} & \rho_{i0} < \rho_{js} &(\textrm{sphere}~i~\textrm{inside~sphere}~j)\\
\textrm{VI} & \textrm{otherwise} &(\textrm{sphere}~j~\textrm{inside~sphere}~j)\\
\end{array}\right.
\end{equation}
and the values of $H_{ij}$ are
\begin{equation}
H_{ij} = \left\{
\begin{array}{r l}
\textrm{0} & 0 \\
\textrm{I} & \frac{1}{8 r_{ij}}\left[ 1 + \frac{2 r_{ij}}{r_{ij}-\rho_{js}} + \frac{1}{r_c^2}\left( r_{ij}^2-4 r_c r_{ij} - \rho_{js}^2\right) +2 \ln \frac{r_{ij}-\rho_{js}}{r_c}\right] \\
\textrm{II} & \frac{\rho_{js}^2}{r_{ij}^2} \frac{\rho_{js}}{r_{ij}^2} \left[ a+ \frac{\rho_{js}^2}{r_{ij}^2}\left( b+\frac{\rho_{js}^2}{r_{ij}^2}\left( c+\frac{\rho_{js}^2}{r_{ij}^2}\left( d+\frac{\rho_{js}^2}{r_{ij}^2} e \right) \right) \right) \right] \\
\textrm{III} & \frac{1}{2} \left[\frac{\rho_{js}}{r_{ij}^2-\rho_{js}^2} + \frac{1}{2 r_{ij}} \ln \frac{r_{ij}-\rho_{js}}{r_{ij}+\rho_{js}} \right]\\
\textrm{IV} & \frac{1}{4} \left[ \frac{1}{\rho_{i0}}\left( 2-\frac{1}{2r_{ij}\rho_{i0}}\left(r_{ij}^2+\rho_{i0}^2-\rho_{js}^2\right)\right) - \frac{1}{r_{ij}+\rho_{js}} + \frac{1}{r_{ij}} \ln \frac{\rho_{i0}}{r_{ij}+\rho_{js}} \right]\\
\textrm{V} & \frac{1}{2} \left[\frac{\rho_{js}}{r_{ij}^2-\rho_{js}^2} + \frac{2}{\rho_{i0}} + \frac{1}{2 r_{ij}} \ln \frac{\rho_{js}-r_{ij}}{r_{ij}+\rho_{js}} \right]\\
\textrm{VI} & 0
\end{array}
\right.
\end{equation}
Below are defined the derivatives of the above functions which are required for force calculations.
%dEij dr
\begin{equation}
\frac{\partial E_{ij}}{\partial r_{ij}} = - k_e \left[
\frac{q_i q_j}{f_{ij}} \frac{\partial D_{ij}}{\partial r_{ij}}  
 - 
 \frac{q_i q_j D_{ij}}{f_{ij}^2} \frac{\partial f_{ij}}{\partial r_{ij}}
\right]
\end{equation}
%d Dij
\begin{equation}
\frac{\partial D_{ij}}{\partial r_{ij}} = \frac{\kappa}{\epsilon_s} \exp{\left(-\kappa f_{ij}\right)\frac{\partial f_{ij}}{\partial r_{ij}}}
\end{equation}
%d f
\begin{equation}
\frac{\partial f_{ij}}{\partial r_{ij}} = \frac{r_{ij}}{f_{ij}} \left[1 - \frac{1}{4} \exp{\left(\frac{-r_{ij}^2}{4 \alpha_i \alpha_j}\right)} \right]
\end{equation}
%d alpha sum
\begin{equation}
\frac{d \alpha_k}{d r_{ij}} = \frac{\alpha_k^2}{\rho_k}\left(1-\textrm{tanh}^2\left(\delta \psi_k - \beta \psi_k^2 + \gamma \psi_k^3\right)\right)
\left( \delta - 2\beta\psi_k+3\beta\psi_k^2\right) \frac{d \psi_k}{d r_{ij}}
\end{equation}
%d psi
\begin{eqnarray}
\frac{d \psi_k}{d r_{ij}}
&=&\rho_{k0} \sum_l \frac{d H_{kl}}{d r_{ij}} \\
&=&\rho_{k0} \sum_l \frac{\partial H_{kl}}{\partial r_{kl}}\frac{d r_{kl}}{d r_{ij}}\\
&=&\rho_{k0} \left[ \frac{\partial H_{kj}}{\partial r_{kj}}\delta_{ki} + \frac{\partial H_{ki}}{\partial r_{ki}}\delta_{kj} \right]
\end{eqnarray}
%d alpha 
\begin{eqnarray}
\frac{d \alpha_k}{d r_{ij}} =
& \frac{\alpha_i^2\rho_{i0}}{\rho_i}\left(1-\textrm{tanh}^2\left(\delta \psi_i - \beta \psi_i^2 + \gamma \psi_i^3\right)\right)
\left( \delta - 2\beta\psi_i+3\beta\psi_i^2\right) \frac{\partial H_{ij}}{\partial r_{ij}} \delta_{ki}\nonumber \\
+ &
\frac{\alpha_j^2\rho_{j0}}{\rho_j}\left(1-\textrm{tanh}^2\left(\delta \psi_j - \beta \psi_j^2 + \gamma \psi_j^3\right)\right)
\left( \delta - 2\beta\psi_j+3\beta\psi_j^2\right) \frac{\partial H_{ji}}{\partial r_{ij}} \delta_{kj}
\end{eqnarray}
\begin{equation}
\frac{\partial E_{ij}}{\partial \alpha_i} = -\frac{1}{\alpha_i}\frac{k_e q_i q_j}{2 f_{ij}^2}\left( \frac{\kappa}{\epsilon_s}\exp{\left(-\kappa f_{ij}\right)} - \frac{D_{ij}}{f_{ij}}\right)
\left(\alpha_i\alpha_j + \frac{r_{ij}^2}{4}\right)\exp{\left(\frac{-r_{ij}^2}{4 \alpha_i \alpha_j}\right)}
\end{equation}
\begin{equation}
\frac{\partial E_{ij}}{\partial \alpha_j} = -\frac{1}{\alpha_j}\frac{k_e q_i q_j}{2 f_{ij}^2}\left( \frac{\kappa}{\epsilon_s}\exp{\left(-\kappa f_{ij}\right)} - \frac{D_{ij}}{f_{ij}}\right)
\left(\alpha_i\alpha_j + \frac{r_{ij}^2}{4}\right)\exp{\left(\frac{-r_{ij}^2}{4 \alpha_i \alpha_j}\right)}
\end{equation}
%d Hij
\begin{equation}
\frac{\partial H_{ij}}{\partial r_{ij}} = \left\{
\begin{array}{r l}
\textrm{0} & 0 \\
\textrm{I} &
\left[-\frac{\left(r_c+\rho_{js}-r_{ij}\right)\left(r_c-\rho_{js}+r_{ij}\right)\left(\rho_{js}^2+r_{ij}^2\right)}{8r_c^2r_{ij}^2\left(\rho_{js}-r_{ij}\right)^2}
-\frac{1}{4r_{ij}^2}\ln \frac{r_{ij}-\rho_{js}}{r_c}
\right] \\
\textrm{II} &
\left[
-4a\frac{\rho_{js}^3}{r_{ij}^5}
-6b\frac{\rho_{js}^5}{r_{ij}^7}
-8c\frac{\rho_{js}^7}{r_{ij}^9}
-10d\frac{\rho_{js}^9}{r_{ij}^{11}}
-12e\frac{\rho_{js}^{11}}{r_{ij}^{13}}
\right] \\
\textrm{III} &
\frac{1}{2}\left[
-\frac{\rho_{js}\left(r_{ij}^2+\rho_{js}^2\right)}{r_{ij}\left(r_{ij}^2-\rho_{js}^2\right)^2}
-\frac{1}{2r_{ij}^2} \ln \frac{r_{ij}-\rho_{js}}{r_{ij}+\rho_{js}}
\right] \\
\textrm{IV} &
\frac{1}{4} \left[
-\frac{1}{2\rho_{i0}^2}
+\frac{r_{ij}^2\left(\rho_{i0}^2-\rho_{js}^2\right)-2 r_{ij}\rho_{js}^3+\rho_{js}^2\left(\rho_{i0}^2-\rho_{js}^2\right)}{2r_{ij}^2\rho_{i0}^2\left(r_{ij}+\rho_{js}\right)^2}
-\frac{1}{r_{ij}^2}\ln\frac{\rho_{i0}}{r_{ij}+\rho_{js}}
\right] \\
\textrm{V} &
\frac{1}{2}\left[
-\frac{\rho_{js}\left(r_{ij}^2+\rho_{js}^2\right)}{r_{ij}\left(r_{ij}^2-\rho_{js}^2\right)^2}
-\frac{1}{2r_{ij}^2} \ln \frac{\rho_{js}-r_{ij}}{r_{ij}+\rho_{js}}
\right] \\
\textrm{VI} & 0
\end{array}
\right.
\end{equation}
Other variables referenced in the above GB equations are
\begin{list}{}
\item $r_{ij}$ - distance between atoms i and j; calculated from atom coordinates.
\item $\kappa$ - debye screening length; calculated from ion concentration, $\kappa^{-1} = \sqrt{\frac{\epsilon_0 \epsilon_p k T}{2 N_A e^2 I}}$; $\kappa^{-1} = 10$~\AA~for 0.1~M monovalent salt.
\item $\epsilon_s$ - dielectric constant of solvent.
\item $\epsilon_p$ - dielectric constant of protein.
\item $\alpha_i$ - Born radius of atom $i$.
\item $\rho_i$ - intrinsic radius of atom $i$ taken from Bondi~\cite{BOND64}.
\item $\rho_0$ - intrinsic radius offset; $\rho_0 = 0.09$~\AA~by default~\cite{ONUF04}.
\item $\rho_{i0} = \rho_i - \rho_0$
\item $\rho_{is} = \rho_{i0} S_{ij}$
\item $S_{ij}$ - atom radius scaling factor~\cite{HAWK96,SRIN99}.
\item $k_e$ - Coulomb's constant, $\frac{1}{4 \pi \epsilon_0}$, 332.063711 kcal~\AA~/ e$^2$.
\item $\{\delta, \beta, \gamma\} = \{0.8, 0, 2.91\}~\textrm{or}~\{1.0, 0.8, 4.85\}$~\cite{ONUF04}
\end{list}


%\subsubsection{3-Phase Calculation}
\subsection{3-Phase Calculation}

The GBIS algorithm requires three phases of calculation, with each phase containing an iteration over all atom pairs with the cutoff.
In phase 1, the screening of atom pairs is summed; at the conclusion of phase 1, the Born radii are calculated.
In phase 2, the $\frac{\partial E_{ij}^{GB}}{\partial r_{ij}}$ force contribution (hereafter called the dEdr force) is calculated as well as the partial derivative of the Born radii with respect to the atom positions, $\frac{d \alpha_i}{d r_{ij}}$.
In phase 3, the $\frac{\partial E_T^{GB}}{\partial \alpha_i}\frac{d \alpha_i}{d r_{ij}}$ force contribution (hereafter called the dEda force) is calculated.


%\subsubsection{Configuration Parameters}
\subsection{Configuration Parameters}

When using GBIS, user's should not use {\tt PME} (because it is not compatible with GBIS).
Periodic boundary conditions are supported but are optional.
User's will need to increase {\tt cutoff}; 16-18~\AA~is a good place to start but user's will have to check their system's behavior and increase {\tt cutoff} accordingly.
GBIS interactions are never excluded regardless of the type of force field used, thus user's can choose any value for {\tt exclude} without affecting GBIS; user's should still choose {\tt exclude} based on the force field as if using explicit solvent.
When using GBIS, multiple timestepping behaves as follows:
the dEdr force is calculated every {\tt nonbondedFreq} steps (as with explicit solvent, 2 is a reasonable frequency) and
the dEda force is calculated every {\tt fullElectFrequency} steps (because dEda varies more slowly than dEdr, 4 is a reasonable frequency).

%begin configuration parameter list
\begin{itemize}

\item
\NAMDCONFWDEF{GBIS}{Use Generalized Born Implicit Solvent?} {{\tt on} or {\tt off}} {{\tt off}} {Turns on GBIS method in NAMD. }

\item
\NAMDCONFWDEF{solventDielectric}{dielectric of water} {positive decimal} {78.5} {Defines the dielectric of the solvent, usually 78.5 or 80.}

\item
\NAMDCONFWDEF{intrinsicRadiusOffset}{shrink the intrinsic radius of atoms (\AA)} {positive decimal} {0.09} {This offset shrinks the intrinsic radius of atoms (used only for calculating Born radius) to give better agreement with Poisson Boltzmann calculations. Most users should not change this parameter.}

\item
\NAMDCONFWDEF{ionConcentration}{concentration of ions in solvent (Molar)} {positive decimal} {0.2} {An ion concentration of 0~M represents distilled water. Increasing the ion concentration increases the electrostatic screening.}

\item
\NAMDCONFWDEF{GBISDelta}{GB$^{OBC}$ parameter for calculating Born radii} {decimal} {1.0} {Use $\{{\tt GBISDelta}, {\tt GBISBeta},{\tt GBISGamma}\} = \{1.0, 0.8, 4.85\}$ for GB$^{OBC}$II and $\{0.8, 0.0, 2.90912\}$ for GB$^{OBC}$I. See $\{\alpha, \beta, \gamma\}$ in \cite{ONUF04} for more information.}

\item
\NAMDCONFWDEF{GBISBeta}{GB$^{OBC}$ parameter for calculating Born radii} {decimal} {0.8} {See {\tt GBISDelta}.}

\item
\NAMDCONFWDEF{GBISGamma}{GB$^{OBC}$ parameter for calculating Born radii} {decimal} {4.85} {See {\tt GBISDelta}.}

\item
\NAMDCONFWDEF{alphaCutoff}{cutoff used in calculating Born radius and derivatives (phases 1 and 3) (\AA)} {positive decimal} {15} {Cutoff used for calculating Born radius. Only atoms within this cutoff de-screen an atom. Though {\tt alphaCutoff} can bet set to be larger or shorter than {\tt cutoff}, since atom forces are more sensitive to changes in position than changes in Born radius, user's should generally set {\tt alphaCutoff} to be shorter than {\tt cutoff}.}

\item
\NAMDCONFWDEF{SASA}{whether or not to calculate SASA} {{\tt on} or {\tt off}} {{\tt off}} {The nonpolar / hydrophobic energy contribution from implicit solvent is calculated; it is proportional to the solvent-accessible surface area (SASA) which is calculated by the Linear Combination of Pairwise Overlap (LCPO) method~\cite{WEIS99}. It evaluated every {\tt nonbondedFreq} steps and its energy is added into the reported {\tt ELECT} energy.}

\item
\NAMDCONFWDEF{surfaceTension}{surface tension of SASA energy} {positive decimal} {0.005~kcal/mol/\AA$^2$} {Surface tension used when calculating hydrophobic {\tt SASA} energy; $E_{\rm nonpolar} = {\rm surfaceTension} \times {\rm surfaceArea}$.}

\end{itemize}

Below is a sample excerpt from a NAMD config file for nonbonded and multistep parameters when using {\tt GBIS} and {\tt SASA}:
\begin{verbatim}
#GBIS parameters
GBIS on
ionConcentration 0.3
alphaCutoff 14
#nonbonded parameters
switching on
switchdist 15
cutoff 16
pairlistdist 18
#hydrophobic energy
sasa on
surfaceTension 0.006
#multistep parameters
timestep 1
nonbondedFreq 2
fullElectFrequency 4
\end{verbatim}
%\bibliographystyle{unsrt}
%\bibliography{ug}
%\bibliography{../../LaTeX/Bibliography/data_base}

\begin{comment}
\end{document}
\end{comment}
